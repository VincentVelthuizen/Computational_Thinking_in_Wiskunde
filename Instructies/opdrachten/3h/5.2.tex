\opdracht[3]In \autoref{foute-abc} zie je een script voor het oplossen van $ax^2+bx+c$. Dit script hoort bij het programma \texttt{Foute-ABC} in je bibliotheek. Wat gaat er fout in dit programma? 

\begin{lstlisting}[language=pseudo, rulesepcolor=\color{opdrachtkleur}, caption={Script van het programma \texttt{Foute-ABC.}}, label={foute-abc}]
ABC Formule
invoer : A, B, C
uitvoer: XL, XR

XL is -B-$\lstsqrt{} (B^{2}-4AC)$ / 2A
XR is -B+$\lstsqrt{} (B^{2}-4AC)$ / 2A
geef XL, XR
\end{lstlisting}

\opdracht[3] \label{abc-formule}%
\? Schrijf een programma dat
\begin{itemize}
	\item De formule $ax^{2}+bx+c$ weergeeft aan de gebruiker en vervolgens vraagt om $a$, $b$ en $c$.
	\item De discriminant uitrekent en weergeeft
	\item Op basis van de discriminant aangeeft hoeveel oplossingen er zijn.
	\item De oplossing geeft als er \'e\'en oplossing is.
	\item De oplossingen geeft als er meerdere oplossingen zijn.
\end{itemize}
\? Controleer je code door te kijken of je de juiste antwoorden krijgt bij opdracht \opdracht{16} en \opdracht{17}.

\opdracht[4] Uit de vergelijking $y=ax^2+bx+c$ kun je nog meer informatie halen. Schrijf nu code waarmee je de co\"ordinaten van de top van de parabool $y=ax^2+bx+c$ kunt weergeven. Je mag bij deze opdracht  zelf kiezen of je hiervoor je programma uit \refopdracht{abc-formule}\  gaat uitbreiden, of dat je een nieuw programma maakt.

